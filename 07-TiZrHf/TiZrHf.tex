\documentclass[hyperref=unicode,presentation,10pt]{beamer}

\usepackage[absolute,overlay]{textpos}
\usepackage{array}
\usepackage{graphicx}
\usepackage{adjustbox}
\usepackage[version=4]{mhchem}
\usepackage{chemfig}
\usepackage{caption}
\usepackage{makecell}

%dělení slov
\usepackage{ragged2e}
\let\raggedright=\RaggedRight
%konec dělení slov

\addtobeamertemplate{frametitle}{
	\let\insertframetitle\insertsectionhead}{}
\addtobeamertemplate{frametitle}{
	\let\insertframesubtitle\insertsubsectionhead}{}

\makeatletter
\CheckCommand*\beamer@checkframetitle{\@ifnextchar\bgroup\beamer@inlineframetitle{}}
\renewcommand*\beamer@checkframetitle{\global\let\beamer@frametitle\relax\@ifnextchar\bgroup\beamer@inlineframetitle{}}
\makeatother
\setbeamercolor{section in toc}{fg=red}
\setbeamertemplate{section in toc shaded}[default][100]

\usepackage{fontspec}
\usepackage{unicode-math}

\usepackage{polyglossia}
\setdefaultlanguage{czech}

\def\uv#1{„#1“}

\mode<presentation>{\usetheme{default}}
\usecolortheme{crane}

\setbeamertemplate{footline}[frame number]

\title[Crisis]
{C2062 -- Anorganická chemie II}

\subtitle{Titan, zirkonium, hafnium a rutherfordium}
\author{Zdeněk Moravec, hugo@chemi.muni.cz \\ \adjincludegraphics[height=60mm]{img/IUPAC_PSP.jpg}}
\date{}

\begin{document}

\begin{frame}
	\titlepage
\end{frame}

\section{Úvod}
\frame{
	\frametitle{}
	\vfill
	\begin{tabular}{|l|l|l|l|}
	\hline
	 & \textit{Titan} & \textit{Zirkonium} & \textit{Hafnium} \\\hline
	 El. k. & 3d$^{2}$ 4s$^{2}$ & 4d$^{2}$ 5s$^{2}$ & 4f$^{14}$ 5d$^{2}$ 6s$^{2}$ \\\hline
	 T$_v$ [$^\circ$C] & 3287 & 4377 & 4603 \\\hline
	 T$_t$ [$^\circ$C] & 1668 & 1855 & 2233 \\\hline
	 Objev & 1791 & 1789 & 1922 \\\hline
	 & šedý\footnote[frame]{Zdroj: \href{https://commons.wikimedia.org/wiki/File:Titan-crystal_bar.JPG}{Alchemist-hp/Commons}} & stříbrnobílý\footnote[frame]{Zdroj: \href{https://commons.wikimedia.org/wiki/File:Zirconium_crystal_bar_and_1cm3_cube.jpg}{Alchemist-hp/Commons}} & ocelově šedé\footnote[frame]{Zdroj: \href{https://commons.wikimedia.org/wiki/File:Hf-crystal_bar.jpg}{Alchemist-hp/Commons}} \\
	 & \begin{minipage}{.2\textwidth}
	 	\adjincludegraphics[width=\linewidth]{img/Titan-crystal_bar.jpg}
	 \end{minipage}
	 	& \begin{minipage}{.2\textwidth}
	 		\adjincludegraphics[width=\linewidth]{img/Zirconium_crystal_bar.jpg}
	 	\end{minipage} & \begin{minipage}{.2\textwidth}
	 	\adjincludegraphics[width=\linewidth]{img/Hf-crystal_bar.jpg} \end{minipage} \\\hline
	\end{tabular}
	\vfill
}

\section{Rutherfordium}
\frame{
	\frametitle{}
	\vfill
	\textbf{Rutherfordium}
	\begin{itemize}
		\item Umělý prvek, transuran, protonové číslo 104, Rf.
		\item Poprvé byl připraven v roce 1964:
		\item \ce{^{242}_{94}Pu + ^{22}_{10}Ne -> ^{264}_{104}Rf}
		\item \ce{^{249}_{98}Cf + ^{12}_{6}C -> ^{257}_{104}Rf + 4 ^1_0n}
		\item Název byl schválen roku 1997.\footnote[frame]{\href{https://doi.org/10.1103/PhysRevLett.31.647}{X-Ray Identification of Element 104}}
		\item Známe izotopy v rozmezí nukleonových čísel od 253 do 270.
		\item Nejstabilnějším izotopem je \ce{^{267}Rf} s poločasem rozpadu 1,3 hodiny.
		\item \ce{^{279}_{110}Ds -> ^{275}_{108}Ds + $\alpha$ -> ^{271}_{106}Sg + $\alpha$}
		\item \ce{^{271}_{106}Sg -> ^{267}_{104}Rf + $\alpha$}
	\end{itemize}

	\begin{tabular}{|l|l|}
		\hline
		\ce{^{261}Rf} & 2,1 s \\\hline
		\ce{^{263}Rf} & 15 minut \\\hline
		\ce{^{265}Rf} & 1,1 minut \\\hline
		\ce{^{267}Rf} & 48 minut \\\hline
	\end{tabular}
	\vfill
}


\frame{
	\frametitle{}
	\vfill
	\begin{columns}
		\begin{column}{.7\textwidth}
			\begin{itemize}
				\item Reakcí s halogeny vznikají tetrahalogenidy \ce{RfX4}.
				\item Ve vodných roztocích jsou Rf$^{4+}$ stabilnější než titaničité sloučeniny, hydrolýza poskytuje ionty RfO$^{2+}$.\footnote[frame]{\href{https://doi.org/10.1351/pac200375010103}{Critical evaluation of the chemical properties of the transactinide elements}}
				\item V 10 M HCl vznikají oktaedrické ionty \ce{RfCl$_6^{2-}$}.\footnote[frame]{\href{https://doi.org/10.1524/ract.2005.93.9-10.519}{Chemical studies on rutherfordium (Rf) at JAERI}}
				\item \ce{Rf^{4+} + 6 Cl^- -> [RfCl6]^{2-}}
				\item Podobné chování bylo pozorováno i při reakci s fluoridy.
			\end{itemize}
		\end{column}
		\begin{column}{.3\textwidth}
			\begin{figure}
				\adjincludegraphics[width=\textwidth]{img/Ernest_Rutherford2.jpg}
				\caption*{Ernest Rutherford.\footnote[frame]{Zdroj: \href{https://commons.wikimedia.org/wiki/File:Ernest_Rutherford2.jpg}{Sadi Carnot/Commons}}}
			\end{figure}
		\end{column}
	\end{columns}
	\vfill
}

\section{Chemické a fyzikální vlastnosti}
\frame{
	\frametitle{}
	\vfill
	\begin{itemize}
		\item Všechny tři kovy jsou poměrně reaktivní, ale na povrchu vytvářejí inertní, kompaktní vrstvu oxidu.
		\item Krystalizují v nejtěsnějším hexagonálním uspořádání.
		\item Mají vysoké teploty tání.
		\item V práškovém stavu jsou pyroforické.
		\item Při zahřívání reagují s většinou nekovů. Titan se také ochotně slučuje s dusíkem. Při teplotě 800~$^\circ$C v dusíkové atmosféře hoří za vzniku nitridu.
		\item \ce{2 Ti + N2 ->[800 $^\circ$C] 2 TiN}
		\item V kyselinách a zásadách se rozpouštějí pouze za horka.
		\item Výjimkou je kyselina fluorovodíková, která vytváří rozpustné fluoro- komplexy.
		\item \ce{2 Ti + 12 HF -> 2 [TiF6]^{3-} + 3 H2 + 6 H+}
	\end{itemize}
	\vfill
}

\frame{
	\frametitle{}
	\vfill
	\begin{itemize}
		\item Vytvářejí stabilní sloučeniny v oxidačním čísle IV.
		\item Díky podobnosti atomových poloměrů jsou titaničité sloučeniny podobné sloučeninám cíničitým.
		\item Sloučeniny v oxidačním čísle III mají redukční účinky.
		\item Sloučeniny \ce{Zr^{III}} a \ce{Hf^{III}} redukují i vodu, proto nemáme poznatky o~chemii jejich vodných roztoků.
		\item V komplexních sloučeninách dosahují koordinačního čísla 8 a někdy i~vyššího.
		\item Chemie titanu a zirkonia je poměrně dobře prozkoumaná, u hafnia je množství poznatků menší.
	\end{itemize}
	\vfill
}

\section{Výskyt a získávání prvků}
\subsection{Titan}
\frame{
	\frametitle{}
	\vfill
	\begin{itemize}
		\item Titan patří mezi hojně rozšířené prvky, jeho koncentrace v zemské kůře je 6320~ppm (9. nejrozšířenější prvek v zemské kůře).
		\item Známe skoro 400 minerálů titanu, nejdůležitějšími jsou ilmenit a rutil.\footnote[frame]{\href{https://www.mindat.org/element/Titanium}{The mineralogy of Titanium}}
		\item Má pět stabilních izotopů: \ce{^{46}Ti}, \ce{^{47}Ti}, \ce{^{48}Ti}, \ce{^{49}Ti} a \ce{^{50}Ti}.
	\end{itemize}
	\begin{columns}
		\begin{column}{.2\textwidth}
			\begin{tabular}{|l|r@{,}l|}
				\hline
				\ce{^{46}Ti} & 8 & 25 \% \\\hline
				\ce{^{47}Ti} & 7 & 44 \% \\\hline
				\ce{^{48}Ti} & 73 & 72 \% \\\hline
				\ce{^{49}Ti} & 5 & 41 \% \\\hline
				\ce{^{50}Ti} & 5 & 18 \% \\\hline
			\end{tabular}
		\end{column}
		\begin{column}{.6\textwidth}
		\begin{figure}
			\adjincludegraphics[height=.35\textheight]{img/Titanium_products.jpg}
			\caption*{Kovový titan.\footnote[frame]{Zdroj: \href{https://commons.wikimedia.org/wiki/File:Titanium_products.jpg}{Mark Fergus/Commons}}}
		\end{figure}
		\end{column}
	\end{columns}
	\vfill
}

\frame{
	\frametitle{}
	\vfill
	\begin{columns}
		\begin{column}{.65\textwidth}
			\begin{itemize}
				\item \textbf{Ilmenit}
				\item \ce{FeTiO3}, trigonální minerál.\footnote[frame]{\href{http://mineraly.sci.muni.cz/oxidy/ilmenit.html}{Ilmenit}}
				\item Často obsahuje také příměsi hořčíku a manganu, \ce{(Fe,Mg,Mn,Ti)O3}.\footnote[frame]{\href{http://www.webmineral.com/data/Ilmenite.shtml}{Ilmenite Mineral Data}}
				\item Struktura obsahuje ionty \ce{Fe^{2+}} a \ce{Ti^{4+}} umístěné v mřížce korundového typu.
				\item V Česku se nachází v Orlických horách, Železných horách, v okolí Třebíče, atd.
				\item Byl nalezen i na Měsíci.\footnote[frame]{\href{https://www.nytimes.com/2015/12/29/science/new-type-of-rock-is-discovered-on-moon.html}{New Type of Rock Is Discovered on Moon}}
			\end{itemize}
		\end{column}
		\begin{column}{.35\textwidth}
			\begin{figure}
				\adjincludegraphics[width=.9\textwidth]{img/Ilmenit-Struktur.png}
				\caption*{Struktura ilmenitu.\footnote[frame]{Zdroj: \href{https://commons.wikimedia.org/wiki/File:Ilmenit-Struktur.png}{Speedpera/Commons}}}
			\end{figure}
		\end{column}
	\end{columns}
	\vfill
}

\frame{
	\frametitle{}
	\vfill
	\begin{columns}
		\begin{column}{.5\textwidth}
			\begin{figure}
				\adjincludegraphics[height=.6\textheight]{img/Ilmenite-173863.jpg}
				\caption*{Ilmenit, Kanada.\footnote[frame]{Zdroj: \href{https://commons.wikimedia.org/wiki/File:Ilmenite-173863.jpg}{Modris Baum/Commons}}}
			\end{figure}
		\end{column}
		\begin{column}{.5\textwidth}
			\begin{figure}
				\adjincludegraphics[height=.6\textheight]{img/Ilmenite-Cassiterite-Quartz-130337.jpg}
				\caption*{Ilmenit, kasiterit a křemen, Namíbie.\footnote[frame]{Zdroj: \href{https://commons.wikimedia.org/wiki/File:Ilmenite-Cassiterite-Quartz-130337.jpg}{Robert M. Lavinsky/Commons}}}
			\end{figure}
		\end{column}
	\end{columns}
	\vfill
}

\frame{
	\frametitle{}
	\vfill
	\begin{columns}
		\begin{column}{.65\textwidth}
			\begin{itemize}
				\item \textbf{Rutil}\footnote[frame]{\href{http://mineraly.sci.muni.cz/oxidy/rutil.html}{Rutil}}
				\item \ce{TiO2}, tetragonální minerál.
				\item Hlavní surovina pro výrobu oxidu titaničitého.
				\item Jedna ze tří přírodních forem oxidu titaničitého.
				\item Je to termodynamicky nejstabilnější forma \ce{TiO2}.
				\item V Česku ho nacházíme u Písku, Golčova Jeníkova, Ledče nad Sázavou, atd.\footnote[frame]{\href{https://www.youtube.com/watch?v=12TWLGP3zNY}{Rutil z Chřenovic}}
			\end{itemize}
		\end{column}

		\begin{column}{.4\textwidth}
			\begin{figure}
				\adjincludegraphics[height=.2\textheight]{img/Hematite-Rutile-57088.jpg}
				\caption*{Rutil a hematit, Švýcarsko.\footnote[frame]{Zdroj: \href{https://commons.wikimedia.org/wiki/File:Hematite-Rutile-57088.jpg}{Robert M. Lavinsky/Commons}}}
			\end{figure}
		\begin{figure}
			\adjincludegraphics[height=.2\textheight]{img/Rutile_et_hématite.jpg}
			\caption*{Rutil a hematit, Brazílie.\footnote[frame]{Zdroj: \href{https://commons.wikimedia.org/wiki/File:Rutile_et_hématite_(Brésil).jpg}{Parent Géry/Commons}}}
		\end{figure}
		\end{column}
	\end{columns}
	\vfill
}

\frame{
	\frametitle{}
	\vfill
	\begin{itemize}
		\item \textbf{Anatas}\footnote[frame]{\href{https://www.mindat.org/min-213.html}{Anatase}}
		\item \ce{TiO2}, čtverečný minerál.
		\item Jedna ze tří přírodních forem oxidu titaničitého.
		\item Občas se využívá jako drahý kámen.
	\end{itemize}
	\begin{columns}
		\begin{column}{.5\textwidth}
			\begin{figure}
				\adjincludegraphics[height=.4\textheight]{img/Anatase_Oisans.jpg}
				\caption*{Anatas, Francie.\footnote[frame]{Zdroj: \href{https://commons.wikimedia.org/wiki/File:Anatase_Oisans.jpg}{Didier Descouens/Commons}}}
			\end{figure}
		\end{column}
		\begin{column}{.5\textwidth}
			\begin{figure}
				\adjincludegraphics[height=.4\textheight]{img/Anatase-mun05-15a.jpg}
				\caption*{Anatas, Balúčistán.\footnote[frame]{Zdroj: \href{https://commons.wikimedia.org/wiki/File:Anatase-mun05-15a.jpg}{Robert M. Lavinsky/Commons}}}
			\end{figure}
		\end{column}
	\end{columns}
	\vfill
}

\frame{
	\frametitle{}
	\vfill
	\begin{itemize}
		\item \textbf{Brookit}\footnote[frame]{\href{https://www.mindat.org/min-787.html}{Brookite}}
		\item \ce{TiO2}, kosočtverečný minerál.
		\item Jedna ze tří přírodních forem oxidu titaničitého.
	\end{itemize}
	\begin{columns}
		\begin{column}{.5\textwidth}
			\begin{figure}
				\adjincludegraphics[height=.4\textheight]{img/Brookite.jpg}
				\caption*{Brookit a křemen, Pákistán.\footnote[frame]{Zdroj: \href{https://commons.wikimedia.org/wiki/File:Brookite,_quartz_11.jpeg}{Parent Géry/Commons}}}
			\end{figure}
		\end{column}
		\begin{column}{.5\textwidth}
			\begin{figure}
				\adjincludegraphics[height=.4\textheight]{img/Brookite-238977.jpg}
				\caption*{Brookit, Pákistán.\footnote[frame]{Zdroj: \href{https://commons.wikimedia.org/wiki/File:Brookite-238977.jpg}{Robert M. Lavinsky/Commons}}}
			\end{figure}
		\end{column}
	\end{columns}
	\vfill
}

\frame{
	\frametitle{}
	\vfill
	\begin{itemize}
		\item \textbf{Perovskit}\footnote[frame]{\href{http://mineraly.sci.muni.cz/oxidy/perovskit.html}{Perovskit}}
		\item \ce{CaTiO3}, rombický minerál.
		\item Struktura obsahuje oktaedry \ce{TiO6} a kubooktaedry \ce{CaO12}.
		\item Syntetické perovskity se využívají ke konstrukci fotovoltaických článků.
	\end{itemize}
	\begin{columns}
		\begin{column}{.5\textwidth}
			\begin{figure}
				\adjincludegraphics[height=.3\textheight]{img/Perovskite-155026.jpg}
				\caption*{Perovskit, USA.\footnote[frame]{Zdroj: \href{https://commons.wikimedia.org/wiki/File:Perovskite-155026.jpg}{Robert M. Lavinsky/Commons}}}
			\end{figure}
		\end{column}
		\begin{column}{.5\textwidth}
			\begin{figure}
				\adjincludegraphics[height=.3\textheight]{img/Kristallstruktur_Perovskit.png}
				\caption*{Struktura perovskitu.\footnote[frame]{Zdroj: \href{https://commons.wikimedia.org/wiki/File:Kristallstruktur_Perovskit.png}{Orci/Commons}}}
			\end{figure}
		\end{column}
	\end{columns}
	\vfill
}

\frame{
	\frametitle{}
	\vfill
	\begin{itemize}
		\item Při výrobě titanu není možné využít redukci uhlíkem, protože dochází ke vzniku velmi stabilních karbidů.
		\item Redukcí kovy, např. Na, Ca nebo Mg zase nedojde k úplnému odstranění kyslíku z výchozího materiálu.
		\item Další překážkou je i vysoká reaktivita titanu za vyšších teplot.
		\item Do 40. let 20. století se pro přípravu titanu využíval postup vyvinutý novozélandským metalurgem Matthewem A. Hunterem v roce 1910.\footnote[frame]{\href{https://doi.org/10.1021/ja01921a006}{Metallic titanium}}
		\item Je založen na redukci chloridu titaničitého kovovým sodíkem v inertní atmosféře.
		\item \ce{TiCl4 + 4 Na -> Ti + 4 NaCl}
		\item Toto byl první postup, který umožnil přípravu kovového titanu bez znečištění nitridy.
	\end{itemize}
	\vfill
}

\frame{
	\frametitle{}
	\vfill
	\begin{itemize}
		\item V roce 1940 tento postup nahrazen ekonomičtějším \textit{Krollovým procesem}, který vyvinul lucembruský metalurg William J. Kroll.\footnote[frame]{\href{https://www.youtube.com/watch?v=oWyrzZh3We0}{Titanium: Kroll Method}}
		\item Místo sodíku probíhá redukce taveninou hořčíku v atmosféře argonu:
		\item \ce{TiCl4 + 2 Mg ->[950 $^\circ$C, Ar] Ti + 2 MgCl2}
		\item Chlorid titaničitý je získáván zahříváním ilmenitu nebo rutilu s koksem v proudu chloru:\footnote[frame]{\href{https://dx.doi.org/10.22063/poj.2017.1453}{Production of titanium tetrachloride (\ce{TiCl4}) from titanium ores: A review}}
		\item \ce{2 FeTiO3 + 7 Cl2 + 6 C ->[900 $^\circ$C] 2 TiCl4 + 2 FeCl3 + 6 CO}
		\item \ce{TiO2 + 2 Cl2 + 2 C ->[1000 $^\circ$C] TiCl4 + 2 CO}
	\end{itemize}
	\vfill
}

\frame{
	\frametitle{}
	\vfill
	\begin{itemize}
		\item Připravený chlorid titaničitý se čistí frakční destilací, tím dojde k odstranění chloridu železitého a dalších nečistot.
		\item Následně se redukuje taveninou hořčíku (T$_{\textrm{t}}$ = 650~$^\circ$C) v uzavřeném reaktoru v atmosféře argonu.
		\item \ce{TiCl4 + 2 Mg ->[950-1150 $^\circ$C, Ar] Ti + 2 MgCl2}
		\item Vzniklý chlorid hořečnatý a přebytečný hořčík se odstraní rozpuštěním ve vodě a zředěné kyselině chlorovodíkové.
		\item Redukce hořčíkem se ukázala jako výhodnější než redukce vápníkem, po té zůstává velký podíl oxidických nečistot.
	\end{itemize}
	\vfill
}

\frame{
	\frametitle{}
	\vfill
	\begin{itemize}
		\item Krollovým procesem získáme tzv. \textit{titanovou houbu}, která se musí dále zpracovat na ingoty.
		\item Houba se rozdrtí na prášek, který se přečistí vyluhováním v lučavce královské (\ce{HNO3 + HCl}).
		\item Poté se přetaví ve vakuu nebo argonové atmosféře za vzniku ingotů.
		\item Ty lze dále zpracovávat na titanové plechy.
		\item Krollův proces lze snadno upravit na redukci sodíkem, která neposkytuje titanovou houbu, ale granulovaný titan.
		\item Další výhodou redukce sodíkem je vznik chloridu sodného, který se dá velmi snadno odstranit z produktu.
		\item \ce{TiCl4 + 4 Na -> Ti + 4 NaCl}
		\item Tento proces je stále velmi neekonomický, proto se neustále hledají levnější metody přípravy čistého titanu.
	\end{itemize}
	\vfill
}

\frame{
	\frametitle{}
	\vfill
	\begin{figure}
		\adjincludegraphics[height=.68\textheight]{img/Titanium-cylinder.jpg}
		\caption*{Titanová houba.\footnote[frame]{Zdroj: \href{https://commons.wikimedia.org/wiki/File:Titanium-cylinder.jpg}{Jurii/Commons}}}
	\end{figure}
	\vfill
}

\frame{
	\frametitle{}
	\vfill
	\begin{itemize}
		\item \textbf{FFC Cambridge Process} -- Fray-Farthing-Chen
		\item Byl vyvinut v Cambridgi, patentován byl v roce 1998.\footnote[frame]{\href{https://www.asminternational.org/documents/10192/1884362/amp16202p051.pdf/c40e8850-2fc7-456b-a0ec-b4b6e650e9bd}{Exploiting the FFC Cambridge Process}}
		\item Elektrolytická metoda redukce oxidu na taveninu kovu.
		\item Metoda je založena na elektrolýze taveniny oxidu kovu, vzniká tavenina čistého kovu.
		\item Elektrolýza probíhá při teplotě 550--850~$^\circ$C, jako rozpouštědlo se využívá zpravidla chlorid vápenatý (T$_{\textrm{t}}$ = 772~$^\circ$C).
		\item V případě titanu získáme houbovitý produkt.
		\item Tuto metodu lze využít i jako alternativní cestu pro výrobu:
		\begin{itemize}
			\item Supravodičů: \ce{Nb3Sn}, \ce{NbTi}
			\item Permanentních magnetů: \ce{Nd-Fe-B}, \ce{Sm-Co}
			\item Katalyzátorů: Pt, Pd, $\ldots$
		\end{itemize}
	\end{itemize}
	\vfill
}

\subsection{Zirkonium}
\frame{
	\frametitle{}
	\vfill
	\begin{itemize}
		\item Koncentrace zirkonia v zemské kůře je 162~ppm.
		\item Známe více než 100 minerálů zirkonia, hlavními jsou zirkon a baddeleyit.\footnote[frame]{\href{https://www.mindat.org/element/Zirconium}{The mineralogy of Zirconium}}
		\item Má čtyři stabilní izotopy: \ce{^{90}Zr}, \ce{^{91}Zr}, \ce{^{92}Zr} a \ce{^{94}Zr}.
	\end{itemize}
		\begin{columns}
		\begin{column}{.2\textwidth}
			\begin{tabular}{|l|r@{,}l|}
				\hline
				\ce{^{90}Zr} & 51 & 45 \% \\\hline
				\ce{^{91}Zr} & 11 & 22 \% \\\hline
				\ce{^{92}Zr} & 17 & 15 \% \\\hline
				\ce{^{94}Zr} & 17 & 38 \% \\\hline
			\end{tabular}
		\end{column}
		\begin{column}{.6\textwidth}
			\begin{figure}
				\adjincludegraphics[height=.4\textheight]{img/Zirconium_crystal_bar.jpg}\caption*{Kovové zirkonium.\footnote[frame]{Zdroj: \href{https://commons.wikimedia.org/wiki/File:Zirconium_crystal_bar_and_1cm3_cube.jpg}{Alchemist-hp/Commons}}}
			\end{figure}
		\end{column}
	\end{columns}
	\vfill
}

\frame{
	\frametitle{}
	\vfill
	\begin{figure}
		\adjincludegraphics[height=.65\textheight]{img/Zirconium_mineral_concentrates_-_world_production_trend.png}
		\caption*{Světová produkce koncentrátu minerálů zirkonu.\footnote[frame]{Zdroj: \href{https://commons.wikimedia.org/wiki/File:Zirconium_mineral_concentrates_-_world_production_trend.svg}{Leyo/Commons}}}
	\end{figure}
	\vfill
}

\frame{
	\frametitle{}
	\vfill
	\begin{itemize}
		\item \textbf{Zirkon},\footnote[frame]{\href{https://mineraly.sci.muni.cz/nesosilikaty/zirkon.html}{Zirkon}} \ce{ZrSiO4}, tetragonální minerál.\footnote[frame]{\href{https://www.mindat.org/min-4421.html}{Zircon}}
		\item Patří do skupiny nesosilikátů\footnote[frame]{\href{https://mineralogie.sci.muni.cz/kap_7_9_nesosil/kap_7_9_nesosil.htm}{Nesosilikáty mají ve své struktuře izolované tetraedry \ce{SiO4}, které jsou v prostoru propojeny přes koordinační polyedry jiných kationtů}} a jde o jeden z hlavních zdrojů pro výrobu kovového zirkonia.
	\end{itemize}
	\begin{columns}
		\begin{column}{.5\textwidth}
			\begin{figure}
				\adjincludegraphics[height=.33\textheight]{img/Zircon-59684.jpg}
				\caption*{Krystal zirkonu, Afganistán.\footnote[frame]{Zdroj: \href{https://commons.wikimedia.org/wiki/File:Zircon-59684.jpg}{Robert M. Lavinsky/Commons}}}
			\end{figure}
		\end{column}
		\begin{column}{.5\textwidth}
			\begin{figure}
				\adjincludegraphics[height=.33\textheight]{img/Aegirine-Zircon-Feldspar-Group-238900.jpg}
				\caption*{Krystaly zirkonu, Malawi.\footnote[frame]{Zdroj: \href{https://commons.wikimedia.org/wiki/File:Aegirine-Zircon-Feldspar-Group-238900.jpg}{Robert M. Lavinsky/Commons}}}
			\end{figure}
		\end{column}
	\end{columns}
	\vfill
}

\frame{
	\frametitle{}
	\vfill
	\begin{itemize}
		\item \textbf{Baddeleyit}
		\item \ce{ZrO2}, monoklinický minerál.\footnote[frame]{\href{https://www.mindat.org/min-480.html}{Baddeleyite}}
	\end{itemize}
	\begin{columns}
		\begin{column}{.5\textwidth}
			\begin{figure}
				\adjincludegraphics[height=.45\textheight]{img/Baddeleyite-md12a.jpg}
				\caption*{Krystaly badeleyitu.\footnote[frame]{Zdroj: \href{https://commons.wikimedia.org/wiki/File:Baddeleyite-md12a.jpg}{Robert M. Lavinsky/Commons}}}
			\end{figure}
		\end{column}
		\begin{column}{.5\textwidth}
			\begin{figure}
				\adjincludegraphics[height=.45\textheight]{img/BaddeleyiteStructure.png}
				\caption*{Krystalová struktura badeleyitu.\footnote[frame]{Zdroj: \href{https://commons.wikimedia.org/wiki/File:BaddeleyiteStructure.png}{Materialscientist/Commons}}}
			\end{figure}
		\end{column}
	\end{columns}
	\vfill
}

\frame{
	\frametitle{}
	\vfill
	\begin{itemize}
		\item Zirkonium se vyrábí, stejně jako titan, Krollovým procesem.
		\item \ce{ZrCl4 + 2 Mg -> Zr + 2 MgCl2}
		\item Pokud potřebujeme zirkonium beze stop kyslíku a dusíku, používá se tzv. \textit{jodová metoda} nebo \textit{van Arkelova-de Boerova metoda}.
		\item Tato metoda se používá pro přípravu nejen zirkonia, ale i velmi čistého titanu, hafnia a vanadu.
		\item Jedinou podmínkou pro použití této metody je existence těkavého halogenidu daného kovu.
		\item Surový materiál je zahříván ve vakuu s parami halogenu.
		\item Vzniklý těkavý halogenid je veden na žhavené wolframové vlákno, kde se rozkládá za vzniku čistého kovu.
		\item Nečistoty zůstávají v pevné fázi v zásobníku.
	\end{itemize}
	\vfill
}

\frame{
	\frametitle{}
	\vfill
	\begin{figure}
		\adjincludegraphics[height=.67\textheight]{img/Van-Arkel-de-Boer-Apparat.png}
		\caption*{1 - přívod vakua; 2 - molybdenová elektroda; 3 - molybdenová síť; 4 - zásobník surového materiálu; 5 - wolframové vlákno\footnote[frame]{Zdroj: \href{https://commons.wikimedia.org/wiki/File:Van-Arkel-de-Boer-Apparat.png}{Roland Mattern/Commons}}}
	\end{figure}
	\vfill
}

\frame{
	\frametitle{}
	\vfill
	\begin{itemize}
		\item Zirkonium standardně obsahuje malý podíl hafnia. Což při běžných aplikacích nevadí.
		\item Problém je ale v jaderných aplikacích, kde se zirkonium používá k potahování palivových tyčí.
		\item Slitina zirkonia s cínem má pro tyto účely ideální vlastnosti, včetně nízké absorpce tepelných neutronů.
		\item Absorpční vlastnosti hafnia jsou až šestsetkrát vyšší, proto je nutné jej odstranit.
		\item Využívá se extrakce dusičnanů tributylfosfátem nebo thiokyanatanů v methylisobutylketonu. Tak je možné snížit obsah hafnia až pod 100~ppm.
		\item Další možností je provést separaci zirkonia zonální tavbou.
	\end{itemize}
	\vfill
}

\subsection{Hafnium}
\frame{
	\frametitle{}
	\vfill
	\begin{itemize}
		\item Koncentrace hafnia v zemské kůře je jen 5,8~ppm.
		\item Vyskytuje se v minerálech společně ze zirkoniem, zpravidla v nižší koncentraci.\footnote[frame]{\href{https://www.mindat.org/element/Hafnium}{The mineralogy of Hafnium}}
		\item To je dáno podobným iontovým poloměrem.
		\begin{itemize}
			\item Iontový poloměr zirkonia pro k.č. 6 je 72~pm.
			\item Iontový poloměr hafnia pro k.č. 6 je 71~pm.
		\end{itemize}
	\item Přírodní hafnium se skládá z pěti stabilních izotopů (\ce{^{176}Hf}, \ce{^{177}Hf}, \ce{^{178}Hf}, \ce{^{179}Hf} a \ce{^{180}Hf}) a jednoho radioizotopu.
	\end{itemize}
	\begin{columns}
		\begin{column}{.5\textwidth}
			\begin{tabular}{|l|r@{,}l|l|}
				\hline
				\ce{^{174}Hf} & 0 & 16 \% & 7,0$\times$10$^{16}$ let \\
				\hline
				\ce{^{176}Hf} & 5 & 26 \% & stabilní \\
				\hline
				\ce{^{177}Hf} & 18 & 60 \% & stabilní\\\hline
				\ce{^{178}Hf} & 27 & 28 \% & stabilní\\\hline
				\ce{^{179}Hf} & 13 & 62 \% & stabilní\\\hline
				\ce{^{180}Hf} & 35 & 08 \% & stabilní\\\hline
			\end{tabular}
		\end{column}
		\begin{column}{.5\textwidth}
		\begin{figure}
			\adjincludegraphics[height=.3\textheight]{img/Hafnium_ebeam_remelted.jpg}
			\caption*{Kovové hafnium.\footnote[frame]{Zdroj: \href{https://commons.wikimedia.org/wiki/File:Hafnium_ebeam_remelted.jpg}{Alchemist-hp/Commons}}}
		\end{figure}
		\end{column}
	\end{columns}
	\vfill
}

\frame{
	\frametitle{}
	\vfill
	\begin{itemize}
		\item Hafnium se také vyrábí Krollovým procesem.
		\item \ce{HfCl4 + 2 Mg ->[1100 $^\circ$C] Hf + 2 MgCl2}
		\item Další čištění lze provést jodovou metodou:
		\item \ce{Hf + 2 I2 ->[500 $^\circ$C] HfI4}
		\item \ce{HfI4 ->[W, 1700 $^\circ$C] Hf + 2 I2}
		\item Po nehodě jaderné elektrárny Fukušima došlo ke snížení poptávky po zirkoniu, což způsobilo zvýšení ceny hafnia.\footnote[frame]{\href{https://www.kitco.com/ind/Albrecht/2015-03-11-Weak-Zirconium-Demand-Depleting-Hafnium-Stock-Piles.html}{Weak Zirconium Demand Depleting Hafnium Stock Piles}}
	\end{itemize}
	\begin{figure}
		\adjincludegraphics[height=.25\textheight]{img/Hafnium_pellets_with_a_thin_oxide_layer.jpg}
		\caption*{Hafnium. Barvy jsou způsobeny tenkým oxidickým filmem na povrchu.\footnote[frame]{Zdroj: \href{https://commons.wikimedia.org/wiki/File:Hafnium_pellets_with_a_thin_oxide_layer.jpg}{Alchemist-hp/Commons}}}
	\end{figure}
	\vfill
}

\section{Využití prvků}
\subsection{Titan}
\frame{
	\frametitle{}
	\vfill
	\begin{itemize}
		\item Kovový titan je za laboratorní teploty odolný vůči korozi a je lehký ($\rho$ = 4,51~g.cm$^{-3}$) a mechanicky odolný.
		\item Proto se využívá ve slitinách pro letecký průmysl.
		\item Vysoké odolnosti těchto slitin vůči mořské vodě se využívá v zařízeních pro odsolování vody.
		\item Slitiny titanu se také využívají pro výrobu kuchyňského nádobí, sportovních potřeb a další aplikace.
		\item Supravodivé dráty NbTi se využívají v MRI (Magnet Resonance Imaging) magnetech a také v magnetech u LHC. Kritická teplota je okolo 10~K a hodnota kritického pole dosahuje 15~T.\footnote[frame]{\href{https://doi.org/10.1016/0011-2275(87)90057-9}{Emergence of Nb-Ti as supermagnet material}}
	\end{itemize}
	\vfill
}

\frame{
	\frametitle{}
	\vfill
	\textbf{Nitinol}
	\begin{columns}
		\begin{column}{.7\textwidth}
			\begin{itemize}
				\item Nitinol je skupina slitin titanu a niklu, zastoupení obou prvků se pohybuje okolo 50~\%.
				\item Hmotnostní zlomek niklu se označuje číslem v názvu, např. \textit{Nitinol60}.
				\item Tyto slitiny vykazují \textit{tvarovou paměť} a \textit{superelasticitu}.
				\item Pokud ho mechanicky deformujeme, zachová si nový tvar. Po zahřátí nad teplotu přechodu, dojde k návratu do původního tvaru.
				\item Toho lze využít ke konstrukci různých pohyblivých součástek, např. servomotorů.\footnote[frame]{\href{https://www.youtube.com/watch?v=sscoMtJV0uY}{Very strong Nitinol Engine running on warm water and ice}}
				\item NASA testuje nitinol jako možný materiál pro bezvzduchové pneumatiky.\footnote[frame]{\href{https://technology.nasa.gov/patent/LEW-TOPS-99}{Superelastic Tire (LEW-TOPS-99)}}
			\end{itemize}
		\end{column}
		\begin{column}{.3\textwidth}
			\begin{figure}
				\adjincludegraphics[width=\textwidth]{img/Nitinol_draht.jpg}
				\caption*{Dráty z nitinolu.\footnote[frame]{Zdroj: \href{https://commons.wikimedia.org/wiki/File:Nitinol_draht.jpg}{Petermaerki/Commons}}}
			\end{figure}
		\end{column}
	\end{columns}
	\vfill
}

\subsection{Zirkonium}
\frame{
	\frametitle{}
	\vfill
	\begin{columns}
		\begin{column}{.7\textwidth}
			\begin{itemize}
				\item Kovové zirkonium je vysoce odolné vůči korozi.
				\item Za vysoké teploty dokáže pohlcovat i stopová množství kyslíku, proto se využívá v tzv. zirkoniových getterech pro deoxygenaci plynů.
				\item \ce{Zr + O2 -> ZrO2}
				\item Palivové tyče pro jaderné reaktory se potahují ochranou vrstvou kovového zirkonia.\footnote[frame]{\href{https://aktualne.cvut.cz/stalo-se/20241111-vedec-z-cvut-vyvinul-efektivnejsi-a-odolnejsi-palivo-pro-jaderne-elektrarny}{Vědec z ČVUT vyvinul efektivnější a odolnější palivo pro jaderné elektrárny}}
				\item Sloučeniny zirkonia se využívají také v medicíně, najdeme je v zubních implantátech, kloubních náhradách, apod.
				\item \ce{ZrO2} je velmi odolný keramický materiál využívaný např. v rotorcích pro MAS NMR.
			\end{itemize}
		\end{column}
		\begin{column}{.3\textwidth}
				\begin{figure}
				\adjincludegraphics[height=.6\textheight]{img/Rotor_ssNMR.png}
				\caption*{Rotor pro MAS NMR.\footnote[frame]{Zdroj: \href{https://commons.wikimedia.org/wiki/File:Rotor_ssNMR.png}{AndrijMahun/Commons}}}
			\end{figure}
		\end{column}
	\end{columns}
	\vfill
}

\frame{
	\frametitle{}
	\vfill
	\begin{figure}
		\adjincludegraphics[height=.6\textheight]{img/Zr-getter.jpg}
		\caption*{Zirkoniové gettery, nový a použitý.}
	\end{figure}
	\vfill
}

\frame{
	\frametitle{}
	\vfill
	\begin{itemize}
		\item Izotop \ce{^{89}Zr} se využívá při diagnostice nádorů pomocí PET (pozitronové emisní tomografie).\footnote[frame]{\href{https://doi.org/10.3390/ijms21124309}{Current Perspectives on $^{89}$Zr-PET Imaging}}
		\item Jeho poločas rozpadu je 78,41 hodin.
		\item Připravuje se ozařováním tenké yttriové fólie protony nebo deuterony:
		\item \ce{^{89}Y + p -> ^{89}Zr + n}
		\item \ce{^{89}Y + ^2H -> ^{89}Zr + 2 n}
		\item Po ozařování je yttrium rozpuštěno v kyselině chlorovodíkové a peroxidu vodíku.
		\item Zirkoničité ionty jsou odděleny od yttria na ionexu a následně jsou eulovány roztokem kyseliny šťavelové.
	\end{itemize}
	\vfill
}

\frame{
	\frametitle{}
	\vfill
	\textbf{Zieglerovy-Nattovy katalyzátory}
	\begin{itemize}
		\item Výroba polyethylenu se původně prováděla za vysokých teplot a tlaků.
		\item Německý chemik Karl Ziegler zjistil, že v přítomnosti směsi \ce{TiCl4} a~\ce{AlEt3} probíhá polymerace za laboratorní teploty a tlaku.
		\item Italský chemik Giulio Natta dokázal vhodnou kombinací katalyzátorů řídit polymeraci propylenu tak, aby vznikal isotaktický nebo syndiotaktický polypropylen.
		\item V roce 1963 získali oba za tento výzkum Nobelovu cenu za chemii.\footnote[frame]{\href{https://www.nobelprize.org/prizes/chemistry/1963/summary/}{The Nobel Prize in Chemistry 1963}}
		\item Kromě \ce{TiCl4} lze využít i \ce{TiCl3} a cyklopentadienyly zirkonia.
	\end{itemize}
	\begin{figure}
		\adjincludegraphics[width=.8\textwidth]{img/ZNonSingleSite.png}
		\caption*{Zjednodušený mechanismus polymerace ethylenu.\footnote[frame]{Zdroj: \href{https://commons.wikimedia.org/wiki/File:ZNonSingleSite.png}{Smokefoot/Commons}}}
	\end{figure}
	\vfill
}

\subsection{Hafnium}
\frame{
	\frametitle{}
	\vfill
	\begin{itemize}
		\item Vzhledem k vysoké schopnosti absorpce neutronů se využívá hafnium v řídících tyčích pro jaderné reaktory.\footnote[frame]{\href{http://large.stanford.edu/courses/2011/ph241/grayson1/}{Control Rods in Nuclear Reactors}}
		\item Slitiny hafnia s dalšími kovy (Ti, Fe, Nb, Ta) se využívají v kosmických technologiích pro konstrukci trysek motorů na kapalné paliva.\footnote[frame]{\href{https://samario01.cbmm.com.br/cgs/publico/VisualizaArquivoBVPublica.ashx?DOC_Codigo=747}{Niobium Alloys and High Temperature Applications}}
		\item Např. slitina C103 byla využita v roce 1965 v programu Apollo.\footnote[frame]{\href{https://www.bayvillechemical.net/niobium-c-103-alloy}{Niobium C-103 Alloy}}
	\end{itemize}
	\begin{center}
		\begin{tabular}{|l|l|}
		\hline
		\textbf{Slitina} & \textbf{Složení} \\\hline
		C103 & 90 \% Nb, 10 \% Hf a 1 \% Ti \\\hline
		FS85 & 61 \% Nb, 10 \% W, 28 \% Ta a 1 \% Zr \\\hline
		Cb129Y & 79.8 \% Nb, 10 \% W, 10 \% Hf a 0.2 \% Y \\\hline
		Cb752 & 87.5 \% Nb, 10 \% W a 2.5 \% Zr \\\hline
		Nb1Zr & 99 \% Nb a 1 \% Zr \\\hline
	\end{tabular}
	\end{center}
	\vfill
}

\section{Sloučeniny}
\subsection{Halogenidy}
\frame{
	\frametitle{}
	\begin{tabular}{|c|c|r@{,}l|r@{,}l|}
		\hline
		 & \textbf{Barva} & \multicolumn{2}{c|}{Teplota tání [$^\circ$C]} & \multicolumn{2}{c|}{Teplota varu [$^\circ$C]} \\\hline
		\ce{TiF4} & bílá & 284 & 0 & \multicolumn{2}{c|}{--} \\\hline
		\ce{TiCl4} & bezbarvý & $-$24 & 0 & 136 & 6 \\\hline
		\ce{TiBr4} & oranžová & 39 & 0 & 230 & 0 \\\hline
		\ce{TiI4} & tmavě hnědá & 150 & 0 & 377 & 0 \\\hline
		\hline
		\ce{TiF3} & fialová & 1200 & 0 & 1400 & 0 \\\hline
		\ce{TiCl3} & červenofialová & 425 & 0 & 960 & 0 \\\hline
		\ce{TiBr3} & modročervená & \multicolumn{2}{c|}{--} & \multicolumn{2}{c|}{--} \\\hline
		\ce{TiI3} & červenofialová & \multicolumn{2}{c|}{--} & \multicolumn{2}{c|}{--} \\\hline
		\hline
		\ce{TiCl2} & černá & 1035 & 0 & 1500 & 0 \\\hline
		\ce{TiBr2} & černá & 500 & 0 & \multicolumn{2}{c|}{--} \\\hline
		\ce{TiI2} & černá & \multicolumn{2}{c|}{--} & \multicolumn{2}{c|}{--} \\\hline
	\end{tabular}
}

\frame{
	\frametitle{}
	\begin{columns}
		\begin{column}{0.75\textwidth}
			\vfill
			\begin{itemize}
				\item Titan vytváří tři stechiometrické chloridy.
				\item \textbf{Chlorid titaničitý}, \ce{TiCl4}, je bezbarvá kapalina. Vzdušnou vlhkostí hydrolyzuje za vzniku oxidu titaničitého:
				\item \ce{TiCl4 + 2 H2O -> TiO2 + 4 HCl}
				\item Využívá se pro výrobu kovového titanu a oxidu titaničitého.
				\item Při kontaktu s vlhkým vzduchem produkuje intenzivní dým, proto se využívá v dýmovnicích a pro tvorbu kouřové clony.
				\item S tetrahydrofuranem vytváří žlutý krystalický solvát, \ce{TiCl4.2THF}.\footnote[frame]{\href{https://doi.org/10.1002/9780470132524.ch31}{Tetrahydrofuran Complexes of Selected Early Transition Metals}}
				\item S objemnějšími ligandy vytváří pětikoordinované komplexy \ce{TiCl4.L}.
			\end{itemize}
			\vfill
		\end{column}
		\begin{column}{0.3\textwidth}
			\begin{figure}
				\adjincludegraphics[width=0.8\textwidth]{img/TiCl4.jpg}
				\caption*{Chlorid titaničitý.\footnote[frame]{Zdroj: \href{https://commons.wikimedia.org/wiki/File:Sample_of_Titanium_tetrachloride_01.jpg}{Σ64/Commons}}}
			\end{figure}
		\end{column}
	\end{columns}
}

\frame{
	\frametitle{}
	\begin{columns}
		\begin{column}{0.85\textwidth}
			\vfill
			\begin{itemize}
				\item \textbf{Chlorid titanitý}, \ce{TiCl3}, je červenofialová krystalická látka.
				\item Známe čtyři polymorfní modifikace a několik krystalických hydrátů.
				\item Připravuje se redukcí chloridu titaničitého hliníkem, takto získáme adukt s \ce{AlCl3}, který je komerčně dostupný:
				\item \ce{3 TiCl4 + Al -> 3TiCl3.AlCl3}
				\item Využívá se jako katalyzátor při výrobě polyolefinů.
				\item \textbf{Chlorid titanatý}, \ce{TiCl2}, je černá krystalická látka.
				\item Krystaly mají vrstevnatou strukturu, stejně jako \ce{CdI2}, tzn. že titan je koordinován šesti chloridy.
				\item Je to silné redukční činidlo, z vody uvolňuje vodík.
				\item Připravuje se termickou disproporcionací \ce{TiCl3}:
				\item \ce{2 TiCl3 ->[500 $^\circ$C] TiCl2 + TiCl4}
			\end{itemize}
			\vfill
		\end{column}
		\begin{column}{0.2\textwidth}
			\begin{figure}
				\adjincludegraphics[width=0.9\textwidth]{img/TiCl3.jpg}
				\caption*{Roztok chloridu titanitého.\footnote[frame]{Zdroj: \href{https://commons.wikimedia.org/wiki/File:TiCl3.jpg}{W. Oelen/Commons}}}
			\end{figure}
		\end{column}
	\end{columns}
}

\frame{
	\frametitle{}
	\vfill
	\begin{itemize}
		\item Halogenidy zirkoničité a hafničité se připravují přímou reakcí z prvků.
		\item V pevném stavu tvoří polymerní struktury tvořené tetraedry propojenými stranami, v plynném stavu jsou tvořeny tetraedrickými molekulami.
	\end{itemize}

	\begin{tabular}{|l|l|l||l|l|l|}
		\hline
		\ce{ZrX4} & barva & T$_t$ [$^\circ$C] &  \ce{HfX4} & barva & T$_t$ [$^\circ$C] \\\hline
		\ce{ZrF4} & bílý & 910 &  \ce{HfF4} & bílý & 970  \\\hline
		\ce{ZrCl4} & bílý & 437 & \ce{HfCl4} & bílý & 432  \\\hline
		\ce{ZrBr4} & bílý & 450 & \ce{HfBr4} & bílý & 424 \\\hline
		\ce{ZrI4} & oranžový & 499 & \ce{HfI4} & červeno-oranžový & 449 \\\hline
	\end{tabular}
	\vfill
}

\subsection{Oxidy a hydroxidy}
\frame{
	\frametitle{}
	\vfill
	\begin{itemize}
		\item \textbf{Oxid titaničitý -- \ce{TiO2}}.
		\item V přírodě se vyskytuje ve dvou minerálech -- rutilu\footnote[frame]{\href{https://mineraly.sci.muni.cz/oxidy/rutil.html}{Rutil}} a~anatasu.\footnote[frame]{\href{https://mineraly.sci.muni.cz/oxidy/anatas.html}{Anatas}}
		\item Hlavní surovinou pro jeho výrobu je minerál ilmenit (\ce{FeTiO3}).
		\item \ce{FeTiO3 + H2SO4 -> FeSO4 + TiO2 + H2O}
		\item Hlavní aplikací je \textit{titanová běloba} -- bílý pigment.
		\item Dále se využívá v kosmetice jako pigment a opalovací krém. (P25)\footnote[frame]{\href{https://www.acsmaterial.com/titanium-dioxide-tio2.html}{Titanium Dioxide (TiO2) P25}}
		\item Perspektivní využití do budoucna je jako povrchová úprava skleněných ploch.
		\item V nanoskopické podobě má fotokatalytické vlastnosti.\footnote[frame]{\href{https://doi.org/10.1021/cr5001892}{Understanding \ce{TiO2} Photocatalysis: Mechanisms and Materials}}
	\end{itemize}
	\vfill
}

\frame{
	\frametitle{}
	\begin{columns}
		\begin{column}{0.5\textwidth}
			\begin{figure}
				\adjincludegraphics[width=.9\textwidth]{img/Rutile_crystal_structure.png}
				\caption*{Struktura rutilu.\footnote[frame]{Zdroj: \href{https://commons.wikimedia.org/wiki/File:Rutile_crystal_structure.png}{Cynthia Striley/Commons}}}
			\end{figure}
		\end{column}
		\begin{column}{0.5\textwidth}
			\begin{figure}
				\adjincludegraphics[width=.9\textwidth]{img/Anatase_crystal_structure.png}
				\caption*{Struktura anatasu.\footnote[frame]{Zdroj: \href{https://commons.wikimedia.org/wiki/File:Anatase_crystal_structure.png}{Cynthia Striley/Commons}}}
			\end{figure}
		\end{column}
	\end{columns}
}

\frame{
\frametitle{}
\vfill
\begin{itemize}
	\item Oxid titaničitý se vyrábí dvěma způsoby.
	\item \textit{Chloridový způsob} je založen na reakci rutilu s chlorem a koksem, kdy vzniká těkavý chlorid titaničitý:\footnote[frame]{\href{https://ti-cons.com/TC2/images/Ti-Cons/pdf/ti-cons_cp_en.pdf}{\ce{TiO2} Chloride Process}}.
	\item \ce{TiO2 + 2 Cl2 + 2 C ->[900 $^\circ$C] TiCl4 + 2 CO}
	\item Chlorid je poté spalován v kyslíkové atmosféře, čímž zároveň dochází k regeneraci chloru:
	\item \ce{TiCl4 + O2 ->[900-1400 $^\circ$C] TiO2 + 2 Cl2}
	\item Tato metoda není vhodná pro rudy obsahující velké množství železa, protože ze vzniklých chloridů není možné získat zpět použitý chlor.
	\item V dnešní době jde o preferovaný způsob výroby \ce{TiO2}.
\end{itemize}
\vfill
}

\frame{
	\frametitle{}
	\vfill
	\begin{itemize}
		\item \textit{Síranový způsob} je založen na reakci mletého ilmenitu s kyselinou sírovou:
		\item \ce{FeTiO3 + 2 H2SO4 ->[80-100 $^\circ$C] FeSO4 + TiOSO4 + 2 H2O}
		\item Vznikající síran železitý je redukován železnýma hoblinama.
		\item Zelená skalice je po ochlazení odfiltrována.
		\item Síran titanylu je hydrolyzován varem ve vodě za vzniku gelu \ce{TiO2}:
		\item \ce{TiOSO4 + n H2O -> TiO2.(n-1)H2O + H2SO4}
		\item Kyselina sírová je neutralizována uhličitanem vápenatým za vzniku sádrovce jako vedlejšího produktu.
		\item Přečištěný gel je kalcinován při teplotě 800--1000~$^\circ$C.
		\item Po kalcinaci je produkt mlet na jemný prášek pomocí tryskového mlýnu.
		\item Pomletý materiál se separuje pomocí flotace.
		\item Tímto postupem získáme anatas, pokud chceme připravit rutil je nutné použít krystalizační zárodek rutilu.
	\end{itemize}
	\vfill
}

\frame{
	\frametitle{}
	\begin{columns}
		\begin{column}{0.6\textwidth}
			\vfill
			\begin{itemize}
				\item Povrchy, které zabraňují usazování nečistot a baktérií.
				\item Tři základní mechanismy:
				\begin{itemize}
					\item Superhydrofobicita
					\item Superhydrofilita
					\item Fotokatalýza
				\end{itemize}
				\item Tyto povrchy nacházíme i v přírodě.
				\begin{itemize}
					\item Lotosový květ
					\item Motýlí křídla
					\item Žraločí kůže
				\end{itemize}
			\end{itemize}
			\vfill
		\end{column}
		\begin{column}{0.4\textwidth}
			\begin{figure}
				\adjincludegraphics[width=.8\textwidth]{img/One_World_Trade_Center_and_7_World_Trade_Center,_Manhattan,_July_2019.jpg}
				\caption*{Mrakodrapy na Manhattanu.\footnote[frame]{Zdroj: \href{https://commons.wikimedia.org/wiki/File:A627,_One_World_Trade_Center_and_7_World_Trade_Center,_Manhattan,_July_2019.jpg}{Brian W. Schaller/Commons}}}
			\end{figure}
		\end{column}
	\end{columns}
}

\frame{
	\frametitle{}
	\vfill
	\begin{itemize}
		\item \textbf{Superhydrofóbní povrchy}.
		\item Silně odpuzují vodu, kapky vody nesmáčí povrch, ale sklouzávají po něm.
		\item Kontaktní úhel ($\theta$, vymezen rozhraním plyn/povrch a plyn/kapalina) je větší než 150°.
		\begin{itemize}
			\item $\cos \theta = \frac{\gamma_{SG} - \gamma_{SL}}{\gamma_{LG}}$, $\gamma$ -- povrchové napětí
		\end{itemize}
		\item Příkladem mohou být fluorované polymery.\footnote[frame]{\href{https://doi.org/10.1002/0471440264.pst594}{Superhydrophobic Polymers}}
		\item Hydrofobicitu povrchu je možné zvýšit pomocí působení plazmatu.
	\end{itemize}
	\begin{figure}
		\adjincludegraphics[width=0.5\textwidth]{img/PEOTF-F4.png}
	\end{figure}
	\vfill
}

\frame{
	\frametitle{}
	\begin{columns}
		\begin{column}{0.5\textwidth}
			\begin{figure}
				\adjincludegraphics[width=\textwidth]{img/DropConnectionAngel.jpg}
				\caption*{Kapka vody na hydrofobním povrchu.\footnote[frame]{Zdroj: \href{https://commons.wikimedia.org/wiki/File:DropConnectionAngel.jpg}{Na2jojon/Commons}}}
			\end{figure}
		\end{column}
		\begin{column}{0.5\textwidth}
			\begin{figure}
				\adjincludegraphics[width=\textwidth]{img/Rame-hart_goniometer.jpg}
				\caption*{Stanovení kontaktního úhlu.\footnote[frame]{Zdroj: \href{https://commons.wikimedia.org/wiki/File:Rame-hart_goniometer.jpg}{Ramehart/Commons}}}
			\end{figure}
		\end{column}
	\end{columns}
	\footnotetext{\href{https://doi.org/10.1002/0471440264.pst594}{Superhydrophobic Polymers}}
}

\frame{
	\frametitle{}
	\begin{columns}
		\begin{column}{0.8\textwidth}
			\vfill
			\begin{itemize}
				\item \textbf{Superhydrofilní povrchy}.
				\item Silně smáčeny vodou.
				\item Kontaktní úhel se blíží 0°.
				\item Superhydrofilicita byla objevena v roce 1995 v japonskou společností \textit{Toto Ltd.}
				\item Výhodou těchto povrchů je snadné čištění olejových skvrn, lze je spláchnout vodou.
				\item Příkladem je oxid titaničitý (\ce{TiO2}) po aktivaci UV složkou slunečního záření.
			\end{itemize}
			\vfill
		\end{column}
		\begin{column}{0.3\textwidth}
			\begin{figure}
				\adjincludegraphics[width=\textwidth]{img/Video_contact_angle.png}
				\caption*{Kapka vody na hydrofilním povrchu.\footnote[frame]{Zdroj: \href{https://commons.wikimedia.org/wiki/File:Video_contact_angle.gif}{Deglr6328/Commons}}}
			\end{figure}
		\end{column}
	\end{columns}
}

\frame{
	\frametitle{}
	\vfill
	\begin{itemize}
		\item \textbf{Katalyzátory} jsou látky, které mění reakční mechanismus a tím i~rychlost probíhající reakce.
		\item Katalyzátor není během reakce spotřebováván, i když se reakce účastní. Proto není nutné používat stechiometrické množství.
		\item Rozlišujeme homogenní a heterogenní katalýzu.
		\item Reakce: \ce{X + Y -> Z} může v přítomnosti katalyzátoru \textbf{C} probíhat takto:
	\end{itemize}
	\begin{columns}
		\begin{column}{0.3\textwidth}
			\begin{tabular}{r@{\ce{->}}l}
				X + \textbf{C} & X\textbf{C} \\
				X\textbf{C} + Y & XY\textbf{C} \\
				XY\textbf{C} & Z\textbf{C} \\
				Z\textbf{C} & Z + \textbf{C} \\
			\end{tabular}
		\end{column}
		\begin{column}{0.7\textwidth}
			\begin{figure}
				\adjincludegraphics[width=0.7\textwidth]{img/CatalysisScheme.png}
				\caption*{Reakční koordináta nekatalyzované a katalyzované reakce.\footnote[frame]{Zdroj: \href{https://commons.wikimedia.org/wiki/File:CatalysisScheme.png}{Smokefoot/Commons}}}
			\end{figure}
		\end{column}
	\end{columns}
	\vfill
}

\frame{
	\frametitle{}
	\vfill
	\begin{itemize}
		\item \textbf{Fotochemie} je obor chemie, který se zabývá vlivem záření na průběh chemických reakcí.
		\item V přírodě se setkáváme např. s fotosyntézou vyvolanou slunečním zářením, v laboratoři zpravidla používáme umělé zdroje záření v oblastech:
		\begin{itemize}
			\item UV (100--400 nm)
			\item VIS (400--750 nm)
			\item IR (750--2500 nm)
		\end{itemize}
	\end{itemize}
	\begin{figure}
		\adjincludegraphics[width=0.75\textwidth]{img/Jablonski_Diagram.png}
		\caption*{Jablonskiho diagram.\footnote[frame]{Zdroj: \href{https://commons.wikimedia.org/wiki/File:Jablonski_Diagram.png}{Organicoboist/Commons}}}
	\end{figure}
	\vfill
}

\frame{
	\frametitle{}
	\vfill
	\begin{itemize}
		\item \textbf{Fotokatalyzátor} je látka která katalyzuje fotochemické reakce.
		\item Rozlišujeme opět homogenní a heterogenní katalyzátory.
		\item U samočistících povrchů se setkáváme s heterogenní katalýzou.
		\item Katalyzátor je nanesen na povrch skla a po ozáření sluncem dochází k degradaci nečistot na menší molekuly, které jsou následně odstraněny působením deště nebo větru.
		\item Tento mechanismus využívá kombinaci fotokatalýzy a hydrofobicity povrchu.
	\end{itemize}
	\begin{center}
		\adjincludegraphics[width=0.75\textwidth]{img/Self-cleaning-photocatalyst.png}
	\end{center}
	\vfill
}

\frame{
	\frametitle{}
	\vfill
	\begin{columns}
		\begin{column}{0.75\textwidth}
			\begin{itemize}
				\item \textit{Oxid zirkoničitý}, \ce{ZrO2}, zirkonia, je velmi tvrdá a nereaktivní látka.
				\item Krystaluje v monoklinické soustavě, při vyšší teplotě má strukturu rutilu.
				\item Za teplot nad 2700~$^\circ$C krystaluje v krychlové soustavě, typ fluorit.
				\item Díky jeho vodivosti a tepelné odolnosti se využívali ke konstrukci prvních zdrojů IR záření -- \textit{Nernstových lamp}.
				\item Ve šperkařství se využívá jako náhrada diamantu.
				\item \textit{Oxid hafničitý}, \ce{HfO2}, je bílý izolant, jde o velmi stabilní sloučeninu hafnia.
				\item Krystaluje ve stejné soustavě jako zirkonia.
			\end{itemize}
		\end{column}
		\begin{column}{0.4\textwidth}
			\begin{figure}
				\adjincludegraphics[width=.8\textwidth]{img/CZ_brilliant.jpg}
				\caption*{Broušený \ce{ZrO2}.\footnote[frame]{Zdroj: \href{https://commons.wikimedia.org/wiki/File:CZ_brilliant.jpg}{Gregory Phillips/Commons}}}
			\end{figure}
		\end{column}
	\end{columns}
	\vfill
}

\frame{
	\frametitle{}
	\vfill
	\begin{figure}
		\adjincludegraphics[height=.7\textheight]{img/ZrO2.png}
		\caption*{Krystalová struktura \ce{ZrO2}.\footnote[frame]{Zdroj: \href{https://commons.wikimedia.org/wiki/File:Kristallstruktur_Zirconium(IV)-oxid.png}{Orci/Commons}}}
	\end{figure}
	\vfill
}

\subsection{Soli}
\frame{
	\frametitle{}
	\vfill
	\begin{columns}
		\begin{column}{0.6\textwidth}
			\begin{itemize}
				\item \textit{Síran titanylu}, \ce{TiOSO4}, bílá pevná látka.
				\item Hydrolýzou poskytuje oxid titaničitý.
				\item Má polymerní strukturu, síra je koordinována tetraedricky, titan vytváří oktaedry \ce{TiO6}.\footnote[frame]{\href{http://dx.doi.org/10.21577/0100-4042.20170427}{Titanyl sulphate, an inorganic polymer: structural studies and vibrational assignment}}
			\end{itemize}
		\end{column}
		\begin{column}{0.45\textwidth}
			\begin{figure}
				\adjincludegraphics[width=1.1\textwidth]{img/TiOSO4.png}
				\caption*{Krystalová struktura \ce{TiOSO4}.\footnote[frame]{Zdroj: \href{https://commons.wikimedia.org/wiki/File:72972-ICSD.png}{Smokefoot/Commons}}}
			\end{figure}
		\end{column}
	\end{columns}
	\vfill
}


\subsection{Karbidy}
\frame{
	\frametitle{}
	\vfill
	\begin{columns}
		\begin{column}{0.7\textwidth}
			\begin{itemize}
				\item \textit{Karbid titanu}, TiC, je extrémně tvrdý keramický materiál, podobný karbidu wolframu.
				\item Jeho tvrdost v Mohsově stupnici je 9,0-9,5.\footnote[frame]{Tvrdost diamantu je 10.}
				\item Vytváří kubické, plošně centrované krystaly, podobně jako chlorid sodný.
				\item V přírodě se vyskytuje jako velmi vzácný, tmavě šedý minerál khamrabaevit.\footnote[frame]{\href{https://www.mindat.org/min-2194.html}{Khamrabaevite}}
				\item Jeho hlavní využití je jako součást kompozitních keramik pro řezné nástroje (\textit{CerMet} -- \textit{Ceramic Metal}).\footnote[frame]{\href{https://www.vut.cz/www_base/zav_prace_soubor_verejne.php?file_id=15571}{Cermety a jejich efektivní využití}}
				\item Jeho vysoké tepelné odolnosti se využívá při konstrukci tepelných štítů raketoplánů.
			\end{itemize}
		\end{column}
		\begin{column}{0.4\textwidth}
			\begin{figure}
				\adjincludegraphics[width=\textwidth]{img/TiC.png}
				\caption*{Krystalová struktura \ce{TiC}.\footnote[frame]{Zdroj: \href{https://commons.wikimedia.org/wiki/File:TiC-xtal-3D-vdW.png}{Ben Mills/Commons}}}
			\end{figure}
		\end{column}
	\end{columns}
	\vfill
}

\frame{
	\frametitle{}
	\vfill
	\begin{itemize}
		\item \textit{Karbid zirkonia}, ZrC, je extrémně tvrdý keramický materiál, podobný karbidu wolframu.
		\item Jeho tvrdost v Mohsově stupnici je vyšší než 8,0.
		\item Vytváří kubické, plošně centrované krystaly.
		\item Vyrábí se práškovou metalurgií, slinováním.
		\item V jaderných reaktorech se využívá karbidu zirkonia, který byl zbaven příměsí hafnia.
		\item Povrchové vrstvy ZrC se zpravidla vyrábějí pomocí CVD.\footnote[frame]{\href{https://doi.org/10.1111/j.1551-2916.2007.02253.x}{Deposition Mechanism for Chemical Vapor Deposition of Zirconium Carbide Coatings}}
		\item \ce{Zr + 2 Cl2 -> ZrCl4}
		\item \ce{ZrCl4 + CH4 + H2 -> ZrC}
	\end{itemize}
	\vfill
}

\frame{
	\frametitle{}
	\vfill
	\begin{columns}
		\begin{column}{0.5\textwidth}
			\begin{figure}
				\adjincludegraphics[height=.55\textheight]{img/ZrN-polyhedral.png}
				\caption*{Krystalová struktura \ce{ZrC}.\footnote[frame]{Zdroj: \href{https://commons.wikimedia.org/wiki/File:ZrN-polyhedral.png}{Ktlabe/Commons}}}
			\end{figure}
		\end{column}
		\begin{column}{0.5\textwidth}
			\begin{figure}
				\adjincludegraphics[height=.55\textheight]{img/Zirconium_carbide_ZrC.jpg}
				\caption*{Práškový \ce{ZrC}.\footnote[frame]{Zdroj: \href{https://commons.wikimedia.org/wiki/File:Zirconium_carbide_ZrC.jpg}{Sa123/Commons}}}
			\end{figure}
		\end{column}
	\end{columns}
	\vfill
}

\frame{
	\frametitle{}
	\vfill
	\begin{itemize}
		\item \textit{Karbid hafnia}, HfC, krystaluje ve struktuře NaCl.
		\item Jeho tvrdost je vyšší než 9,0.
		\item Zpravidla vytváří nestechiometrické fáze \ce{HfC_x}, x je mezi 0,5 a 1.
		\item S nižším obsahem uhlíku než 0,8 je paramagnetický, při vyšším obsahu diamagnetický.\footnote[frame]{\href{https://www.assignmentpoint.com/science/chemistry/hafnium-carbide-hfc.html}{Hafnium Carbide (HfC)}}
		\item Má velmi vysokou teplotu tání, $>$3900~$^\circ$C.
		\item Připravuje se redukcí oxidu uhlíkem:
		\item \ce{HfO2 + 2 C ->[2000 $^\circ$C] HfC + CO2}
		\item Na rozdíl od předchozích karbidů je výrazně dražší, proto má velmi omezené použití.
	\end{itemize}
	\vfill
}

\subsection{Nitridy}
\frame{
	\frametitle{}
	\vfill

	\begin{columns}
		\begin{column}{.8\textwidth}
			\begin{itemize}
				\item \textit{Nitrid titanu}, TiN, je extrémně tvrdý materiál.
				\item Využívá se ve formě povrchových vrstev, jako ochrana řezných nástrojů\footnote[frame]{\href{https://www.youtube.com/watch?v=kpCpse6Jvqw}{TiN: The titanium-nitride coating}} nebo z dekorativních důvodů (má zlatou barvu).
				\item Připravuje se reakcí titanu s dusíkem za vysoké teploty, často s~využitím metod PVD nebo CVD.
			\end{itemize}
			\begin{figure}
				\adjincludegraphics[height=.25\textheight]{img/Titanium_nitride_TiN.jpg}
				\caption*{Nitrid titanu.\footnote[frame]{Zdroj: \href{https://commons.wikimedia.org/wiki/File:Titanium_nitride_TiN.jpg}{Sa123/Commons}}}
			\end{figure}
		\end{column}
		\begin{column}{.25\textwidth}
			\begin{figure}
				\adjincludegraphics[height=.6\textheight]{img/Titanium_nitride_coating.jpg}
				\caption*{Vrták s povrchovou úpravou z TiN.\footnote[frame]{Zdroj: \href{https://commons.wikimedia.org/wiki/File:Titanium_nitride_coating.jpg}{Peter Binter/Commons}}}
			\end{figure}
		\end{column}
	\end{columns}
	\vfill
}

\frame{
	\frametitle{}
	\vfill
	\begin{itemize}
		\item \textit{Nitridy zirkonia a hafnia}, jsou také velmi tvrdé materiály, které se využívají i k dekorativním účelům.\footnote[frame]{\href{https://doi.org/10.3390/ma12172728}{Carbides and Nitrides of Zirconium and Hafnium}}
		\item Připravují se buď přímou reakcí z prvků, což je exotermní reakce, probíhající za zvýšené teploty nebo \textit{karbotermální nitridací} z oxidu:
		\item \ce{2 ZrO2 + 4 C + N2 -> 2 ZrN + 4 CO}
		\item \ce{2 HfO2 + 4 C + 2 NH3 -> 2 HfN + 4 CO + 3 H2}
		\item U hafnia známe i dva subnitridy, oba s trigonální strukturou:
		\begin{itemize}
			\item \ce{Hf3N2} -- stabilní do teploty 1970~$^\circ$C
			\item \ce{Hf4N3} -- stabilní do teploty 2300~$^\circ$C
		\end{itemize}
	\end{itemize}

	\begin{figure}
		\adjincludegraphics[height=.25\textheight]{img/Zirconiun_nitride_coated_cutters.jpg}
		\caption*{Frézy s povrchovou vrstvou ZrN.\footnote[frame]{Zdroj: \href{https://commons.wikimedia.org/wiki/File:Zirconiun_nitride_coated_cutters..JPG}{Heilite/Commons}}}
	\end{figure}
	\vfill
}

\input{../Last}

\end{document}